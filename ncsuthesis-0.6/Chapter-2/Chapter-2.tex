\chapter{Background}
\label{chap-two}

\section{Flying Ad-hoc networks (FANETs)}

A wireless ad hoc network is a decentralized network of nodes where there is no specialized equipment like routers/switches/access points to facilitate communication, and the communication medium uses wireless protocols e.g. IEEE 802.11. Since there is no need to set-up any special infrastructure the network can be spun up quickly. Generally, nodes are mobile (hence termed as Mobile ad-hoc network - MANET) and depending on use-cases, node type or node mobility they are further classified as Vehicular ad hoc networks (VANETs) or Flying ad hoc networks (FANETs) or Smart-phone ad hoc networks (SPANs).

Flying ad hoc networks are a subset of MANETs and consist of nodes capable of flight. The nodes have high degree of mobility, and the distance between the FANET nodes is larger as compared to MANET nodes. These nodes can be manned airplanes, unmanned guided drones or unmanned unguided drones. The span of the network also varies from few hundred meters to several kilometers. One important distinction about FANETs is that the participating nodes have a common mission and their mobility model depends on the mission \cite{OUBBATI201729}. In this section, we discuss the different characteristics, applications, advantages and challenges of FANETs.

\subsection{FANET Characteristics and Applications}

\subsubsection{FANET Characteristics}
FANETs differ from MANETs on several characteristics. Some of them are as follows.

\textbf{Node Type and mobility:} Typical nodes in a FANET would be unmanned aerial vehicles (UAVs) which can be further differentiated based on their size, namely, large, small and mini UAVs \cite{OUBBATI201729}. Their mobility, transmission range, altitude, speed, and energy autonomy depends accordingly on their corresponding sizes. A comparison between these types of UAVs is presented in Table \ref{tab:uav_comparison} \cite{OUBBATI201729}.

\begin{table}
\caption{Comparison of different type of FANET nodes}
\label{tab:uav_comparison}
\begin{tabular}{|p{0.21\linewidth}|p{0.21\linewidth}|p{0.24\linewidth}|p{0.24\linewidth}|}
\toprule
Properties & Mini UAVs & Small UAVs & Large UAVs\\
\midrule
Transmission Range & Small & Medium & Large\\
\midrule
Altitude 	&  about 300 m   & about 2000 m  &  > 3000 m \\
\midrule
Speed & Medium & High & Very High \\
\midrule
Energy Autonomy & Restricted (Battery Dependent) & Not Restricted (On-board solar panels or fuel) &  Not Restricted (On-board solar panels or fuel) \\
\bottomrule
\end{tabular}
\end{table}

\textbf{Network Connectivity and QoS requirements:} Depending on the UAV density (number of UAVs is a unit volume) and the node mobility, the connectivity varies. Although the node movement is predefined according to the mission, it can be updated during the mission due to sensor errors, environmental conditions or change of mission altogether. These affect the link quality and efficiency of routing protocols. QoS requirements also differ on a mission basis. Some missions are network delay tolerant while others are not. Depending on the applications, some FANETs can aim to optimize different subsets of QoS parameters like latency, bandwidth, packet loss, jitter etc.


\subsubsection{FANETs applications} 
Application requiring FANET can be grouped into three categories based on the entities in the network and cooperation model among them \cite{BEKMEZCI20131254}.

\textbf{Multi-UAV cooperation:} Certain tasks require the UAVs to autonomously complete the mission in a time limit without human interaction. In such applications, there is no awareness of the ground environment and thereby doesn't need any ground infrastructure. The UAVs cooperate and coordinate according to the predetermined application and correct the flight path in response to the environmental factors and errors. Example applications would include plume wrapping, target detection, monitoring disasters etc. 

\textbf{UAV-to-ground collaboration:} In certain applications, the UAVs need to periodically relay data to a ground location so that it can be acted upon in real time. Such applications require ground infrastructure which can be either mobile or fixed. Examples include search and rescue missions, military monitoring etc.

\textbf{UAV-to-VANET collaboration:} In the third kind, the aerial UAVs can communicate and coordinate with vehicles on the ground to accomplice certain tasks. Such applications may require ground infrastructure and human involvement. This type of hybrid ad hoc network is a relatively new kind of wireless communication. Examples include route guidance, traffic monitoring, data packet delivery etc.

The above information is summarized in Table \ref{tab:uav_applications}

\begin{table}
\caption{UAV Applications}
\label{tab:uav_applications}
\begin{tabular}{|p{0.22\linewidth}|p{0.22\linewidth}|p{0.22\linewidth}|p{0.22\linewidth}|}
\toprule
Parameters & Multi-UAV cooperation & UAV-to-ground tasks & UAV-to-VANET collaborations\\
\midrule
UAV density & High & Medium & Low\\
\midrule
Ground Environment 	& Not aware &  Aware & Aware  \\
\midrule
Infrastructure & No & Yes &  Maybe \\
\midrule
Obstacles effect & Low & High & Medium \\
\midrule
Task Duration & Limited & Not Limited & Not limited \\
\midrule
Human Operator & No & Yes & Maybe \\
\bottomrule
\end{tabular}
\end{table}

\section{Routing in 2D ad hoc networks}

In a large ad hoc network all the nodes might not be in the direct transmission range of each other, hence routing techniques are needed for successful communication between distant nodes. Node mobility and wireless medium leads to frequent topology changes and link failures which makes routing in MANETs a challenging task.

The common MANET routing protocols are generally grouped into two broad groups, namely, topology-based and position-based routing \cite{6238283}. In this section, we will discuss some examples of these protocols and their properties.

\subsection{Topology-based routing protocols}
These approaches are like destination IP address-based routing. Topology-based routing protocols maintain a network graph and a path for each destination. The network graph creation can be further classified as proactive, reactive and hybrid approaches. 

\textbf{Proactive approach:} In this routing approach, a node maintains the path to each node in the network through periodic updates to other nodes. Since the path is already known the packets can be immediately forwarded towards the destination. The downside is that due to high mobility the link information becomes stale quickly which leads to routing failure. The overhead of maintaining a dynamic network makes this approach difficult to scale.

\textbf{Reactive approach} In this routing approach, a node initiates path discovery when it needs to reach a destination. The node sends out route discovery request messages to other nodes (the node might keep the route information for future use) as needed. 

\textbf{Hybrid approach} In hybrid approaches, the network is divided into zones. For intra-zone communications proactive approach is adopted while for inter-zone communications a reactive approach is adopted. 
While topology-based approaches guarantee packet delivery if the destination is reachable, they have a high maintenance cost in terms of memory and communication overhead. Moreover, topology-based approaches are more suitable for relatively static networks compared to MANETs.

\subsection{Position-based routing protocols}
These approaches route packets based on the geographic position of the destination. In highly dynamic networks, position-based routing approaches have been experimentally confirmed to be more scalable than those which don't use geographic information \cite{Stojmenovic:2002:PRA:2288474.2290160}. Geographic routing approaches don't need to share and store link information - rather just the immediate radio neighbors - and thereby make routing decisions on a hop-by-hop basis. Whereas, position-based routing requires the nodes to know their own location, which can be obtained by a location system such a GPS. The accuracy of a GPS might not be enough for multi-UAV systems which are highly mobile at very high speed. Moreover, GPS provides location updates every second which might not be sufficient for several multi-UAV missions. To overcome this issue, the nodes can be equipped with Inertial measurement units, which can be calibrated by GPS signals.

Position-based routing protocols are generally classified on two properties, `location service' and `forwarding strategy'.

\subsubsection{Location Services} \label{loc_service}
Besides their own location, the nodes may need to know the location of destination nodes. Most position-based routing protocols employ a location service to maintain and distribute the location of a node. The location services can be classified based on the number of nodes that maintain the location information and for the number of nodes for which the location servers maintain the information. Thereby, the four possible combinations can be \cite{967595}

\textbf{Some-for-some:} Some nodes in the network maintain and provide the location information for some of the nodes.

\textbf{Some-for-all:} Some nodes in the network maintain and provide the location information for all the nodes.

\textbf{All-for-all:} All nodes maintain the location information for every other node.

\textbf{All-for-some:} All nodes maintain the location information for a subset of the nodes. 
Moreover, there are different methods to distribute location information. A few of the proposed methods are flooding \cite{Basagni:1998:DRE:288235.288254}, grid location service \cite{Li:2000:SLS:345910.345931} and quorum-based location service \cite{769770}.

\subsubsection{Forwarding Strategies}
After a node has obtained the location of the destination node, there needs to be a forwarding/routing strategy to deliver the packet. Forwarding strategies in geographical routing can be grouped as follows:

\textbf{Greedy forwarding} Assuming a source node S needs to send a packet to a destination node D, where S and D are not in the direct transmission range of each other, node S will pick an intermediate node T from among the nodes in the transmission range of node S such that T is the closer to node D than S. Node T will follow the same procedure until a penultimate node which had D in its direct transmission range receives the packet and forwards the packet to D. 
Greedy routing strategies are guaranteed to be loop-free (since the packet never travels backward), however, greedy strategies don't guarantee packet delivery even though there is a path between source S to destination D. It happens when a node can't find an intermediate node which is closer to the destination than the node itself. Nodes in such situation are referred to as concave nodes. There are several variations of greedy forwarding and the recovery strategies presented in the literature \cite{OUBBATI201729} \cite{6238283} \cite{967595} \cite{Stojmenovic:2002:PRA:2288474.2290160}.

\textbf{Face routing:} In this technique, faces on a planar graph are traversed following a technique known as `right hand rule', where the algorithm keeps track of all the times it crosses the line connecting the source to destination. After covering an entire face, the algorithm moves onto one of the intersections that are nearest to the destination. This continues until the destination is eventually reached \cite{4448977} \cite{6238283}. Face routing is also a common recovery strategy in greedy routing strategy. Although the complexity of face routing is higher than greedy routing, face routing guarantees delivery - where possible. Various modifications to the face routing strategy are presented in \cite{6238283}.

\textbf{Hybrid Greedy-Face Routing:} While Greedy routing has less algorithmic complexity, face routing guarantees delivery. Hybrid approaches combine the two approaches i.e. it uses greedy routing while it is possible and then resorts to face routing when it encounters a routing void (local maxima). The primary challenge in this approach is to determine the exact moment when the algorithm should switch back to greedy routing. If the switch is too early, then the algorithm might get stuck in another local maxima, but switching too late will lead to inefficient face routing for too long. The protocols that tackle this problem have been surveyed in \cite{6238283}.

Another approach that doesn't use either one of these approaches was presented in Location Aided Routing \cite{Ko:1998:LRM:288235.288252}. This approach determines a `request zone' and an `expected zone' according to the packet parameters and then forwards the packet in that region.

\section{Routing in 3D ad hoc networks}

FANETs are characterized by high mobility which leads to a frequent change in the underlying terrains. Not only this may introduce frequent topology changes but also terrain-induced blind spots which affect the wireless channel. The wireless channel is also affected by the antenna orientation - which changes frequently due to high mobility - and thereby causes link fluctuations. Moreover, communication requirements in a FANET are mission specific and can widely vary from one type to another. For example, in a search and rescue mission, the information should be transmitted to the base station in real-time while some surveillance missions might choose to gather and store on board and delivering it to the base after the mission has completed or in batches.

These characteristics are peculiar to FANET and hence require an exclusive study of routing protocols which shall be particularly suited to aerial networking. However, most of the proposed geographic routing protocols and techniques have been developed with 2D networks in mind. Thus, these protocols are applicable for networks where the height of the network is less than the transmission radius of the nodes. However, they do not extend easily to 3D networks. For example, face routing would fail to guarantee delivery in 3D networks because planar sub-graph and face perimeters are only applicable to 2D planes \cite{4721268}.

One of the proposed approaches is the Unit Ball Graph \cite{4509730}, which extends the Unit Disk Graph approach. The authors in \cite{4509730} \cite{Durocher2010} have concluded that a deterministic recovery strategy is impossible in 3D networks (as opposed in 2D networks) and hence they have proposed a randomized recovery strategy in their localized geographic routing algorithm. Some other 3D routing protocols have been summarized in \cite{BEKMEZCI20131254} \cite{7437029}.