\chapter{INTRODUCTION}
\label{chap-one}
\section{Aerial networks and the challenges}

\textbf{Content}: Critical operations like environment sensing, disaster monitoring etc. can benefit from multi-uav clusters but there are several challenges critical for their success like real-time inter-drone communication  and autonomous control. 

\textbf{Intent}: Introduce the readers to some operations that would benefit from autonomously operating multi-UAV swarm and establish inter-drone communication as the critical challenge in success of such missions.

The use of unmanned aerial vehicles (UAVs) originated in military operations and has gradually expanded to civilian applications, thanks to the advances in sensors, communication, and embedded systems. Furthermore, the past decade has seen a surge in civil applications of UAVs which has outpaced military deployments, with about a million UAVs being registered with the Federal Aviation Authority \cite{mydroneservices}. The civilian applications of UAVs are in scientific, recreational, agricultural, product delivery, surveillance and many others. The civil use of UAVs has been classified into four broad categories, namely, search and rescue (SAR), coverage, delivery, and construction \cite{7463007}.
Single UAV systems have been in use for decades, but such systems have some challenging issues like limited communication range, bandwidth, dependency on a single system, operational cost etc. On the other hand, mini-UAVs have restricted capabilities in terms of power, sensing, communication, and computation; however, most of these issues can be mitigated by employing a team of multiple UAVs. Therefore, recently, applications which employ a group of small UAVs have gained more interest because of the several associated benefits. Compared to a single large UAV working on a mission, a multi-UAV cluster of drones can be employed for efficient and reliable mission outcome because of the UAV sizes, capabilities, maneuverability, little or no threat to human life and buildings/properties. Moreover, a team of UAVs working together has the potential to perform a task that goes beyond the individual capabilities of a single UAV. 

However, an autonomously operating cluster of UAVs has its own set of challenges like cooperative path planning, collision avoidance, reliable wireless connectivity among each other and to the base station (in some cases), mission specific quality of service (QoS) requirements etc. At the core of all these problems is the challenge to establish a robust inter-network - in an ad-hoc manner - that facilitates reliable and efficient communication and coordination among the UAVs in the team. 

Mobile Ad-hoc NETworks (MANETs) and Wireless Sensor Networks (WSNs) are the traditional ways to set-up a network without any infrastructure. The nodes in the network act as the end-hosts as well as the routers to forward the messages. Setting up such networks is a challenging problem but there is an underlying assumption that the nodes are not highly mobile, and the connecting links have some degree of stability. This assumption is relieved in Vehicular Ad-hoc NETworks (VANETs), where the nodes are highly mobile, however, the nodes in a VANET follow some pattern (e.g. car following model \cite{rothery1992car}) and the movement occurs primarily in two directions. The solutions from WSNs, MANETs, and VANETs do provide some insights on how to proceed with the challenges of link diversity, QoS requirements, and high node mobility but it is still not clear as to whether networking protocols developed for ground networks can be readily deployed in UAV networks \cite{7463007}. 
Ad-hoc networks for UAVs are generally termed as Flying Ad-hoc NETworks (FANETs) and a few of the characteristics that sets them apart from other ad-hoc networks are:
\begin{itemize}
    \item High node mobility in three dimensions 
    \item Heterogenous links: A UAV in a FANETs can be connected to other UAVs (air-to-air) or to a base station (air-to-ground) or even to a satellite (air-to-satellite)
    \item Primary motivation for their existence: `The aerial networks are not just communication networks and as such they have varying yet very specific mission requirements depending on the application' \cite{7463007}
\end{itemize}  

In other words, the requirements from a FANET changes from application to application and even during the same application. Therefore, we need to study and revisit the communication requirements and the protocols at each layer of the inter-networking stack of the aerial networks.

There is a rich literature of work on routing protocols for MANETs. Some of the protocols have proposed to extend the traditional protocols of table-based routing in a proactive (keeping redundant paths or continuously checking on link quality) or reactive (find a route when needed) or hybrid (proactive inside a zone and reactive between zones) manner. Other efforts have proposed routing protocols that make use of geographical locations of the nodes. While geographical routing protocols have been experimentally shown to be more scalable and robust than topology-based protocols for MANETs, geographical routing protocols are still not directly applicable to aerial networks where the height of the network is more than the transmission radius of the nodes. This is because some of the key concepts in 2D geographic routing either do not apply in 3D networks or are computationally expensive \cite{6238283}.

In this thesis, we study the problem of routing packets in an aerial three-dimensional ad-hoc network with highly mobile nodes, where maintaining network topology either proactively or reactively is not feasible. Our approach is to use the inherently broadcast nature of the wireless medium as an advantage by making the intermediate nodes do an `opportunistic' forwarding towards the destination node or region. We have studied the qualitative and quantitative requirements from a network for the plume wrapping and tracking problem \cite{8080382}, as a sample application of autonomous multi-UAV missions. We present an extended version of our previous work on geographic routing that we called `Petal Routing \cite{6133499}' and its applicability to the plume wrapping and tracking mission at hand. We show that our proposed protocol matches the reliability of a network-wide flooding algorithm while significantly reducing the overhead. Finally, we present a formal analysis of the protocol's performance based on metrics like average end-to-end delay, transmission overhead, average number of hops, delivery ratio, the effect of network density.


\section{Organization} 
